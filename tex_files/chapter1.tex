\documentclass[main.tex]{subfiles} 
\begin{document}

\section{Mangfold i naturen}
Ved endt opplæring skal eleven kunne
\begin{itemize}[noitemsep]
\item forklare hovedtrekkene i evolusjonsteorien og gjøre rede for observasjoner som støtter teorien
\item beskrive oppbygningen av dyre- og planteceller og forklare hovedtrekkene i fotosyntese og celleånding
\item gjøre rede for celledeling og for genetisk variasjon og arv
\item forklare hovedtrekk i teorier for hvordan jorda endrer seg og har endret seg gjennom tidene, og grunnlaget for disse teoriene
\item undersøke og registrere biotiske og abiotiske faktorer i et økosystem i nærområdet og forklare sammenhenger mellom faktorene
\item observere og gi eksempler på hvordan menneskelig aktivitet har påvirket et naturområde, undersøke ulike interessegruppers syn på påvirkningen og foreslå tiltak som kan verne naturen for framtidige generasjoner
\item gi varierte eksempler på hvordan samer utnytter ressurser i naturen
\end{itemize}

\subsection{Evolusjonsteorien}

\subsection{Celler og oppbygging av celler}

\subsection{Jordens utvikling}

\subsection{Miljø og naturen}

\end{document}