\documentclass[main.tex]{subfiles} 
\begin{document}

\section{Kropp og helse}
Ved endt opplæring skal eleven kunne
\begin{itemize}[noitemsep]
\item beskrive nervesystemet og hormonsystemet og forklare hvordan de styrer prosesser i kroppen
\item beskrive kort fosterutviklingen og hvordan en fødsel foregår
\item formulere og drøfte problemstillinger knyttet til seksualitet, seksuell orientering, kjønnsidentitet, grensesetting og respekt, seksuelt overførbare sykdommer, prevensjon og abort
\item forklare hvordan egen livsstil kan påvirke helsen, herunder slanking og spiseforstyrrelser, sammenligne informasjon fra ulike kilder, og diskutere hvordan helseskader kan forebygges
\item gi eksempler på samisk og annen folkemedisin og diskutere forskjellen på alternativ medisin og skolemedisin
\end{itemize} 

\subsection{Nerve- og hormonsystemet}

\subsection{Fosterutviklingen}

\subsection{Seksualitet}

\subsection{Livsstil og alternativ behandling}

\end{document}