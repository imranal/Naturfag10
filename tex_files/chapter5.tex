\documentclass[main.tex]{subfiles} 

\begin{document}

\section{Fenomener og stoffer - Fysikk} 
Ved endt opplæring skal eleven kunne
\begin{itemize}[noitemsep]
\item bruke begrepene strøm, spenning, resistans, effekt og induksjon til å forklare resultater fra forsøk med strømkretser
\item forklare hvordan vi kan produsere elektrisk energi fra fornybare og ikke-fornybare energikilder, og diskutere hvilke miljøeffekter som følger med ulike måter å produsere energi på
\item gjøre rede for begrepene fart og akselerasjon, måle størrelsene med enkle hjelpemidler og gi eksempler på hvordan kraft er knyttet til akselerasjon
\item gjøre forsøk og enkle beregninger med arbeid, energi og effekt
\item gjøre greie for hvordan trafikksikkerhetsutstyr hindrer og minsker skader ved uhell og ulykker
\item gjennomføre forsøk med lys, syn og farger, og beskrive og forklare resultatene
\end{itemize}

\subsection{Elektrisk strøm}

\subsection{Fart og Akselerasjon}

\subsection{Arbeid, Energi og Effekt}

\subsection{Lys og dens egenskaper}

\subsection{Trafikksikkerhet}

\end{document}