\documentclass[main.tex]{subfiles} 

\begin{document}

\section{Forskerspiren}
Ved endt opplæring skal eleven kunne
\begin{itemize}[noitemsep]
\item formulere testbare hypoteser, planlegge og gjennomføre undersøkelser av dem og diskutere observasjoner og resultater i en rapport
\item innhente og bearbeide naturfaglige data, gjøre beregninger og framstille resultater grafisk
\item skrive forklarende og argumenterende tekster med referanser til relevante kilder, vurdere kvaliteten ved egne og andres tekster og revidere tekstene
\item forklare betydningen av å se etter sammenhenger mellom årsak og virkning og forklare hvorfor argumentering, uenighet og publisering er viktig i naturvitenskapen
\item identifisere naturfaglige argumenter, fakta og påstander i tekster og grafikk fra aviser, brosjyrer og andre medier, og vurdere innholdet kritisk
\item følge sikkerhetstiltak som er beskrevet i HMS-rutiner og risikovurderinger
\end{itemize}

\subsection{Rapportskriving}

\subsection{Bearbeiding av naturfaglig data}

\subsection{Kildekritikk og argumentering}

\subsection{Sikkerhetstiltak i laben}

\end{document}