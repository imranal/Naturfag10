\documentclass[main.tex]{subfiles} 
\begin{document}

\section*{Forord}
Dette heftet inneholder en oversikt over alt elevene ved 10. trinn i naturfag skal trenge å kunne. Heftet er arrangert etter hvert kompetansemål som elevene skal oppfylle ved slutten av året. Hvert kapittel er utformet etter et sett med kompetansemål som forfatteren mener passer godt sammen. En oversikt over alle kompetansemål er tilgjengelig på udir.no:
\begin{align*}
\text{\url{https://www.udir.no/kl06/NAT1-03/Hele/Kompetansemaal/kompetansemal-etter-10.-arstrinn}}
\end{align*}
Nå følger det nå en oversikt over alle kapitlene i heftet. Kapittel 1 heter \textbf{Mangfold i naturen}. Her skal elevene ved endt opplæring kunne
\begin{itemize}[noitemsep]
\item forklare hovedtrekkene i evolusjonsteorien og gjøre rede for observasjoner som støtter teorien
\item beskrive oppbygningen av dyre- og planteceller og forklare hovedtrekkene i fotosyntese og celleånding
\item gjøre rede for celledeling og for genetisk variasjon og arv
\item forklare hovedtrekk i teorier for hvordan jorda endrer seg og har endret seg gjennom tidene, og grunnlaget for disse teoriene
\item undersøke og registrere biotiske og abiotiske faktorer i et økosystem i nærområdet og forklare sammenhenger mellom faktorene
\item observere og gi eksempler på hvordan menneskelig aktivitet har påvirket et naturområde, undersøke ulike interessegruppers syn på påvirkningen og foreslå tiltak som kan verne naturen for framtidige generasjoner
\item gi varierte eksempler på hvordan samer utnytter ressurser i naturen
\end{itemize} 
Kapittel 2 heter \textbf{Kropp og helse}. Her skal elevene ved endt opplæring kunne
\begin{itemize}[noitemsep]
\item beskrive nervesystemet og hormonsystemet og forklare hvordan de styrer prosesser i kroppen
\item beskrive kort fosterutviklingen og hvordan en fødsel foregår
\item formulere og drøfte problemstillinger knyttet til seksualitet, seksuell orientering, kjønnsidentitet, grensesetting og respekt, seksuelt overførbare sykdommer, prevensjon og abort
\item forklare hvordan egen livsstil kan påvirke helsen, herunder slanking og spiseforstyrrelser, sammenligne informasjon fra ulike kilder, og diskutere hvordan helseskader kan forebygges
\item gi eksempler på samisk og annen folkemedisin og diskutere forskjellen på alternativ medisin og skolemedisin
\end{itemize} 
Kapittel 3, 4, 5 handler om \textbf{Fenomener og stoffer} er fordelt henholdsvis etter følgende kompetansemål. Kapittel 3:
\begin{itemize}[noitemsep]
\item beskrive universet og ulike teorier for hvordan det har utviklet seg
\item undersøke et emne fra utforskingen av verdensrommet, og sammenstille og presentere informasjon fra ulike kilder
\end{itemize}
Kapittel 4:
\begin{itemize}[noitemsep]
\item vurdere egenskaper til grunnstoffer og forbindelser ved bruk av periodesystemet
\item undersøke egenskaper til noen stoffer fra hverdagen og gjøre enkle beregninger knyttet til fortynning av løsninger
\item undersøke og klassifisere rene stoffer og stoffblandinger etter løselighet i vann, brennbarhet og sure og basiske egenskaper
\item planlegge og gjennomføre forsøk med påvisningsreaksjoner, separasjon av stoffer i en blanding og analyse av ukjent stoff
\item undersøke hydrokarboner, alkoholer, karboksylsyrer og karbohydrater, beskrive stoffene og gi eksempler på framstillingsmåter og bruksområder
\item forklare hvordan råolje og naturgass er blitt til
\end{itemize}
Kapittel 5:
\begin{itemize}[noitemsep]
\item bruke begrepene strøm, spenning, resistans, effekt og induksjon til å forklare resultater fra forsøk med strømkretser
\item forklare hvordan vi kan produsere elektrisk energi fra fornybare og ikke-fornybare energikilder, og diskutere hvilke miljøeffekter som følger med ulike måter å produsere energi på
\item gjøre rede for begrepene fart og akselerasjon, måle størrelsene med enkle hjelpemidler og gi eksempler på hvordan kraft er knyttet til akselerasjon
\item gjøre forsøk og enkle beregninger med arbeid, energi og effekt
\item gjøre greie for hvordan trafikksikkerhetsutstyr hindrer og minsker skader ved uhell og ulykker
\item gjennomføre forsøk med lys, syn og farger, og beskrive og forklare resultatene
\end{itemize}
Kapittel 6 heter \textbf{Teknologi og design}, her skal eleven kunne 
\begin{itemize}[noitemsep]
\item utvikle produkter ut fra kravspesifikasjoner og vurdere produktenes funksjonalitet, brukervennlighet og livsløp i forhold til bærekraftig utvikling
\item teste og beskrive egenskaper ved materialer som brukes i en produksjonsprosess, og vurdere materialbruken ut fra miljøhensyn
\item beskrive et elektronisk kommunikasjonssystem, forklare hvordan informasjon overføres fra avsender til mottaker, og gjøre rede for positive og negative konsekvenser
\end{itemize}
Siste kapittel, kapittel 7, heter \textbf{Forskerspiren}. Etter endt opplæring skal eleven kunne:
\begin{itemize}[noitemsep]
\item formulere testbare hypoteser, planlegge og gjennomføre undersøkelser av dem og diskutere observasjoner og resultater i en rapport
\item innhente og bearbeide naturfaglige data, gjøre beregninger og framstille resultater grafisk
\item skrive forklarende og argumenterende tekster med referanser til relevante kilder, vurdere kvaliteten ved egne og andres tekster og revidere tekstene
\item forklare betydningen av å se etter sammenhenger mellom årsak og virkning og forklare hvorfor argumentering, uenighet og publisering er viktig i naturvitenskapen
\item identifisere naturfaglige argumenter, fakta og påstander i tekster og grafikk fra aviser, brosjyrer og andre medier, og vurdere innholdet kritisk
\item følge sikkerhetstiltak som er beskrevet i HMS-rutiner og risikovurderinger
\end{itemize}
For eleven er det viktig å merke seg de forskjellige kompetansenivåene som er krevd ved endt opplæring. Følgende tabell gir en oversikt over kompetansenivå som rangeres fra høyt, middels, og tilslutt lav kompetansenivå.
\begin{table}
\label{tab:kompetansenivå}
\centering
\begin{tabular}{ | m{6cm} | m{6cm}| m{6cm} | m{6cm}} 
\hline
\textbf{Kompetanse på øverste trinn} 
{\color{red}vurdere, kritisere, planlegge, videreutvikle, presisere, produsere} & {\color{red}drøfte, utlede, realisere, styre, justere, utvide} & {\color{red}diskutere, dokumentere, improvisere, kombinere, integrere, forme} & {\color{red} generalisere, trekke slutninger, beherske, påvirke, fornye}\\ 
\hline
\textbf{Kompetanse på mellomste trinn} 
{\color{blue}påvise, forklare, tilpasse, utføre, fortolke} & {\color{blue}gjøre rede for, bruke, forberede, ta initiativ, formulere} & {\color{blue}sammenligne, organisere, forstå, ta ansvar for, løse} & {\color{blue}kommunisere, fortelle, velge, tolerere, beregne} \\ 
\hline
\textbf{Kompetanse på nederste trinn} 
{\color{PineGreen}gjenta, gjenkjenne, gjengi, angi, navngi} & {\color{PineGreen}definere, liste opp, beskrive, skjelne, streke under} & {\color{PineGreen}oppdage, beskrive, føle, iakta, lytte} & {\color{PineGreen}observere, motta inntrykk, følge med, sanse, merke}\\ 
\hline
\end{tabular}
\caption{Kompetansenivå etter Blooms taksonomi}
\end{table}
Det er verdt å ta en kikk på denne tabellen fra gang til gang for å sammenligne kompetansemålene med hvilket nivå som er krevd av deg som elev. Videre i delkapitlene vil en farge kode brukes for å indikere hvilket nivå det forventes av elevene. Høyt kompetansenivå vil markeres med fargen {\color{red}rødt}, mellomtrinn med {\color{blue}blått} og laveste nivå med fargen {\color{PineGreen}furugrønn}.
\end{document}